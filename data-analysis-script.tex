% Options for packages loaded elsewhere
\PassOptionsToPackage{unicode}{hyperref}
\PassOptionsToPackage{hyphens}{url}
%
\documentclass[
]{article}
\usepackage{amsmath,amssymb}
\usepackage{lmodern}
\usepackage{iftex}
\ifPDFTeX
  \usepackage[T1]{fontenc}
  \usepackage[utf8]{inputenc}
  \usepackage{textcomp} % provide euro and other symbols
\else % if luatex or xetex
  \usepackage{unicode-math}
  \defaultfontfeatures{Scale=MatchLowercase}
  \defaultfontfeatures[\rmfamily]{Ligatures=TeX,Scale=1}
\fi
% Use upquote if available, for straight quotes in verbatim environments
\IfFileExists{upquote.sty}{\usepackage{upquote}}{}
\IfFileExists{microtype.sty}{% use microtype if available
  \usepackage[]{microtype}
  \UseMicrotypeSet[protrusion]{basicmath} % disable protrusion for tt fonts
}{}
\makeatletter
\@ifundefined{KOMAClassName}{% if non-KOMA class
  \IfFileExists{parskip.sty}{%
    \usepackage{parskip}
  }{% else
    \setlength{\parindent}{0pt}
    \setlength{\parskip}{6pt plus 2pt minus 1pt}}
}{% if KOMA class
  \KOMAoptions{parskip=half}}
\makeatother
\usepackage{xcolor}
\usepackage[margin=1in]{geometry}
\usepackage{color}
\usepackage{fancyvrb}
\newcommand{\VerbBar}{|}
\newcommand{\VERB}{\Verb[commandchars=\\\{\}]}
\DefineVerbatimEnvironment{Highlighting}{Verbatim}{commandchars=\\\{\}}
% Add ',fontsize=\small' for more characters per line
\usepackage{framed}
\definecolor{shadecolor}{RGB}{248,248,248}
\newenvironment{Shaded}{\begin{snugshade}}{\end{snugshade}}
\newcommand{\AlertTok}[1]{\textcolor[rgb]{0.94,0.16,0.16}{#1}}
\newcommand{\AnnotationTok}[1]{\textcolor[rgb]{0.56,0.35,0.01}{\textbf{\textit{#1}}}}
\newcommand{\AttributeTok}[1]{\textcolor[rgb]{0.77,0.63,0.00}{#1}}
\newcommand{\BaseNTok}[1]{\textcolor[rgb]{0.00,0.00,0.81}{#1}}
\newcommand{\BuiltInTok}[1]{#1}
\newcommand{\CharTok}[1]{\textcolor[rgb]{0.31,0.60,0.02}{#1}}
\newcommand{\CommentTok}[1]{\textcolor[rgb]{0.56,0.35,0.01}{\textit{#1}}}
\newcommand{\CommentVarTok}[1]{\textcolor[rgb]{0.56,0.35,0.01}{\textbf{\textit{#1}}}}
\newcommand{\ConstantTok}[1]{\textcolor[rgb]{0.00,0.00,0.00}{#1}}
\newcommand{\ControlFlowTok}[1]{\textcolor[rgb]{0.13,0.29,0.53}{\textbf{#1}}}
\newcommand{\DataTypeTok}[1]{\textcolor[rgb]{0.13,0.29,0.53}{#1}}
\newcommand{\DecValTok}[1]{\textcolor[rgb]{0.00,0.00,0.81}{#1}}
\newcommand{\DocumentationTok}[1]{\textcolor[rgb]{0.56,0.35,0.01}{\textbf{\textit{#1}}}}
\newcommand{\ErrorTok}[1]{\textcolor[rgb]{0.64,0.00,0.00}{\textbf{#1}}}
\newcommand{\ExtensionTok}[1]{#1}
\newcommand{\FloatTok}[1]{\textcolor[rgb]{0.00,0.00,0.81}{#1}}
\newcommand{\FunctionTok}[1]{\textcolor[rgb]{0.00,0.00,0.00}{#1}}
\newcommand{\ImportTok}[1]{#1}
\newcommand{\InformationTok}[1]{\textcolor[rgb]{0.56,0.35,0.01}{\textbf{\textit{#1}}}}
\newcommand{\KeywordTok}[1]{\textcolor[rgb]{0.13,0.29,0.53}{\textbf{#1}}}
\newcommand{\NormalTok}[1]{#1}
\newcommand{\OperatorTok}[1]{\textcolor[rgb]{0.81,0.36,0.00}{\textbf{#1}}}
\newcommand{\OtherTok}[1]{\textcolor[rgb]{0.56,0.35,0.01}{#1}}
\newcommand{\PreprocessorTok}[1]{\textcolor[rgb]{0.56,0.35,0.01}{\textit{#1}}}
\newcommand{\RegionMarkerTok}[1]{#1}
\newcommand{\SpecialCharTok}[1]{\textcolor[rgb]{0.00,0.00,0.00}{#1}}
\newcommand{\SpecialStringTok}[1]{\textcolor[rgb]{0.31,0.60,0.02}{#1}}
\newcommand{\StringTok}[1]{\textcolor[rgb]{0.31,0.60,0.02}{#1}}
\newcommand{\VariableTok}[1]{\textcolor[rgb]{0.00,0.00,0.00}{#1}}
\newcommand{\VerbatimStringTok}[1]{\textcolor[rgb]{0.31,0.60,0.02}{#1}}
\newcommand{\WarningTok}[1]{\textcolor[rgb]{0.56,0.35,0.01}{\textbf{\textit{#1}}}}
\usepackage{graphicx}
\makeatletter
\def\maxwidth{\ifdim\Gin@nat@width>\linewidth\linewidth\else\Gin@nat@width\fi}
\def\maxheight{\ifdim\Gin@nat@height>\textheight\textheight\else\Gin@nat@height\fi}
\makeatother
% Scale images if necessary, so that they will not overflow the page
% margins by default, and it is still possible to overwrite the defaults
% using explicit options in \includegraphics[width, height, ...]{}
\setkeys{Gin}{width=\maxwidth,height=\maxheight,keepaspectratio}
% Set default figure placement to htbp
\makeatletter
\def\fps@figure{htbp}
\makeatother
\setlength{\emergencystretch}{3em} % prevent overfull lines
\providecommand{\tightlist}{%
  \setlength{\itemsep}{0pt}\setlength{\parskip}{0pt}}
\setcounter{secnumdepth}{-\maxdimen} % remove section numbering
\usepackage{booktabs}
\usepackage{longtable}
\usepackage{array}
\usepackage{multirow}
\usepackage{wrapfig}
\usepackage{float}
\usepackage{colortbl}
\usepackage{pdflscape}
\usepackage{tabu}
\usepackage{threeparttable}
\usepackage{threeparttablex}
\usepackage[normalem]{ulem}
\usepackage{makecell}
\usepackage{xcolor}
\ifLuaTeX
  \usepackage{selnolig}  % disable illegal ligatures
\fi
\IfFileExists{bookmark.sty}{\usepackage{bookmark}}{\usepackage{hyperref}}
\IfFileExists{xurl.sty}{\usepackage{xurl}}{} % add URL line breaks if available
\urlstyle{same} % disable monospaced font for URLs
\hypersetup{
  pdftitle={Exercício de Análise Estatística de Ensaio Clínico},
  pdfauthor={Guilherme Camargo Brito},
  hidelinks,
  pdfcreator={LaTeX via pandoc}}

\title{Exercício de Análise Estatística de Ensaio Clínico}
\usepackage{etoolbox}
\makeatletter
\providecommand{\subtitle}[1]{% add subtitle to \maketitle
  \apptocmd{\@title}{\par {\large #1 \par}}{}{}
}
\makeatother
\subtitle{PUCRS}
\author{Guilherme Camargo Brito}
\date{April 02, 2023}

\begin{document}
\maketitle

{
\setcounter{tocdepth}{2}
\tableofcontents
}
\hypertarget{introduuxe7uxe3o}{%
\subsection{introdução}\label{introduuxe7uxe3o}}

\begin{quote}
Esse relatório é referente a análise estatística de um ensaio clínico
para a disciplina Métodos laboratoriais e modelos experimentais
aplicados à pesquisa do curso de pós-graduação em Biologia Celular e
Molecular da PUCRS.
\end{quote}

Foram utilizados os seguintes métodos estatísticos:

\begin{itemize}
\tightlist
\item
  Análise descritiva: média, mediana, desvio padrão, erro padrão e
  intervalo interquartil.
\item
  Teste de normalidade: Shapiro-Wilk para definição de uso de testes
  paramétricos ou não-paramétricos.
\item
  Análise de frequência: frequência absoluta e relativa.
\item
  Análise de diferenças entre grupos: teste t de duas amostras de Welch
  (variáveis paramétricas) ou Mann-Whitney U (variáveis
  não-paramétricas).
\item
  Análise de diferenças entre grupos de idade (\textgreater=20 vs
  \textless20 anos): Teste t de duas amostras de Welch (variáveis
  paramétricas) ou Mann-Whitney U (variáveis não-paramétricas).
\item
  Cálculo de tamanho de efeito: Cohen's d (variáveis paramétricas) ou
  Coeficiente de correlação ponto-bisserial (variáveis
  não-paramétricas).
\item
  Análise de correlações entre variáveis: correlação produto-momento de
  Pearson (variáveis paramétricas) ou correlação de Spearman (variáveis
  não-paramétricas).
\item
  Todas as análises consideraram um nível alfa de 5\% (p.value = 0.05).
\end{itemize}

\hypertarget{muxe9todos}{%
\subsection{Métodos}\label{muxe9todos}}

\hypertarget{importauxe7uxe3o-e-pre-processamentos-dos-dados}{%
\subsubsection{Importação e Pre-processamentos dos
Dados}\label{importauxe7uxe3o-e-pre-processamentos-dos-dados}}

\begin{quote}
Essa etapa consiste na importação dos dados e na codificação dos
caracteres especiais para ASCII (remove acentuação, ç e outros que
causam erros ao executar o código).
\end{quote}

\begin{itemize}
\tightlist
\item
  Os dados do estudo são importados de uma tabela de excel
  ``data.xlsx'', que deve conter cada unidade observacional em uma linha
  e cada variável em um coluna.
\item
  As variáves de interesse devem ser listadas em arquivos de texto
  abaixo na pasta ``input''.
\item
  Os arquivos determinam quais variaveis serão incluidas em cada teste
  estatistico:

  \begin{itemize}
  \tightlist
  \item
    ``stats.txt'' = análise descritiva
  \item
    ``freqs.txt'' = análise de frequência
  \item
    ``diffs.txt'' = análise de diferenças entre grupos
  \item
    ``corrs.txt'' = análise de correlações entre variáveis
  \item
    ``age\_diffs.txt'' = análise de diferenças pelo efeito da idade,
    \textgreater= ou \textless{} de 20 anos
  \end{itemize}
\end{itemize}

\begin{Shaded}
\begin{Highlighting}[]
\FunctionTok{library}\NormalTok{(tidyverse)}
\FunctionTok{library}\NormalTok{(rio)}
\FunctionTok{library}\NormalTok{(broom)}
\FunctionTok{library}\NormalTok{(openxlsx)}
\FunctionTok{library}\NormalTok{(kableExtra)}

\NormalTok{p.value }\OtherTok{\textless{}{-}} \FloatTok{0.05}

\NormalTok{input }\OtherTok{\textless{}{-}} \FunctionTok{paste0}\NormalTok{(}
    \StringTok{"input/"}\NormalTok{,}
    \FunctionTok{c}\NormalTok{(}
        \StringTok{"stats.txt"}\NormalTok{,}
        \StringTok{"freqs.txt"}\NormalTok{,}
        \StringTok{"diffs.txt"}\NormalTok{,}
        \StringTok{"corrs.txt"}\NormalTok{,}
        \StringTok{"age\_diffs.txt"}
\NormalTok{    )}
\NormalTok{)}

\NormalTok{read }\OtherTok{\textless{}{-}} \ControlFlowTok{function}\NormalTok{(path) \{}
    \FunctionTok{names}\NormalTok{(}\FunctionTok{import}\NormalTok{(path, }\AttributeTok{setclass =} \StringTok{"tibble"}\NormalTok{))}
\NormalTok{\}}

\NormalTok{encoding }\OtherTok{\textless{}{-}} \ControlFlowTok{function}\NormalTok{(data) \{}
    \FunctionTok{lapply}\NormalTok{(data, }\ControlFlowTok{function}\NormalTok{(x) \{}
        \FunctionTok{iconv}\NormalTok{(x, }\AttributeTok{from =} \StringTok{"UTF{-}8"}\NormalTok{, }\AttributeTok{to =} \StringTok{"ASCII//TRANSLIT"}\NormalTok{)}
\NormalTok{    \})}
\NormalTok{\}}

\NormalTok{targets }\OtherTok{\textless{}{-}} \FunctionTok{lapply}\NormalTok{(input, read) }\SpecialCharTok{\%\textgreater{}\%} \FunctionTok{lapply}\NormalTok{(encoding)}
\FunctionTok{names}\NormalTok{(targets) }\OtherTok{\textless{}{-}} \FunctionTok{c}\NormalTok{(}\StringTok{"stats"}\NormalTok{, }\StringTok{"freqs"}\NormalTok{, }\StringTok{"diffs"}\NormalTok{, }\StringTok{"corrs"}\NormalTok{, }\StringTok{"age\_diffs"}\NormalTok{)}

\NormalTok{data }\OtherTok{\textless{}{-}} \FunctionTok{import}\NormalTok{(}\StringTok{"input/data.xlsx"}\NormalTok{, }\AttributeTok{setclass =} \StringTok{"tibble"}\NormalTok{, }\AttributeTok{na =} \StringTok{"NA"}\NormalTok{) }\SpecialCharTok{\%\textgreater{}\%}
    \FunctionTok{setNames}\NormalTok{(}\FunctionTok{iconv}\NormalTok{(}\FunctionTok{colnames}\NormalTok{(.), }\AttributeTok{from =} \StringTok{"UTF{-}8"}\NormalTok{, }\AttributeTok{to =} \StringTok{"ASCII//TRANSLIT"}\NormalTok{)) }\SpecialCharTok{\%\textgreater{}\%}
    \FunctionTok{mutate}\NormalTok{(}\AttributeTok{Categoria\_Idade =} \FunctionTok{if\_else}\NormalTok{(Idade }\SpecialCharTok{\textless{}=} \DecValTok{20}\NormalTok{, }\StringTok{"20\_ou\_menos"}\NormalTok{, }\StringTok{"acima\_de\_20"}\NormalTok{))}
\end{Highlighting}
\end{Shaded}

\hypertarget{teste-de-normalidade}{%
\subsubsection{Teste de Normalidade}\label{teste-de-normalidade}}

Aqui define-se uma função que aplica o teste de Shapiro-Wilk para
verificar se os dados seguem uma distribuição normal. A partir desses
resultados, define-se a aplicação de testes paramétricos ou
não-paramétricos nas análises subsequentes.

\begin{Shaded}
\begin{Highlighting}[]
\NormalTok{shapiro }\OtherTok{\textless{}{-}} \ControlFlowTok{function}\NormalTok{(data, vars) \{}
\NormalTok{    data }\SpecialCharTok{\%\textgreater{}\%}
        \FunctionTok{select}\NormalTok{(}\FunctionTok{all\_of}\NormalTok{(vars)) }\SpecialCharTok{\%\textgreater{}\%}
        \FunctionTok{gather}\NormalTok{(Variável, Valor) }\SpecialCharTok{\%\textgreater{}\%}
        \FunctionTok{group\_by}\NormalTok{(Variável) }\SpecialCharTok{\%\textgreater{}\%}
        \FunctionTok{summarise}\NormalTok{(}
            \AttributeTok{n =} \FunctionTok{n}\NormalTok{(),}
            \AttributeTok{shapiro\_w =} \FunctionTok{shapiro.test}\NormalTok{(Valor)}\SpecialCharTok{$}\NormalTok{statistic,}
            \AttributeTok{shapiro\_p =} \FunctionTok{shapiro.test}\NormalTok{(Valor)}\SpecialCharTok{$}\NormalTok{p.value,}
            \AttributeTok{normality =} \FunctionTok{if\_else}\NormalTok{(}\FunctionTok{shapiro.test}\NormalTok{(Valor)}\SpecialCharTok{$}\NormalTok{p.value }\SpecialCharTok{\textgreater{}}\NormalTok{ p.value,}
                \StringTok{"parametric"}\NormalTok{, }\StringTok{"non{-}parametric"}
\NormalTok{            )}
\NormalTok{        )}
\NormalTok{\}}
\end{Highlighting}
\end{Shaded}

\hypertarget{estatuxedsticas-descritivas}{%
\subsubsection{Estatísticas
Descritivas}\label{estatuxedsticas-descritivas}}

Aqui define-se a função que calcula as estatísticas descritivas para as
variáveis de interesse. Nesse caso Idade, Dias\_ATB,
Dias\_internacao,Altura\_cm ,Peso\_kg, IMC\_absoluto, DNA\_corrigido15x,
VEF1percprevist, CVFpercprevist, SpO2final,VO2mLkgminfinal,
VEabsolutofinal, ReservaVentilatória.

\begin{Shaded}
\begin{Highlighting}[]
\NormalTok{calc\_stats }\OtherTok{\textless{}{-}} \ControlFlowTok{function}\NormalTok{(data, vars, }\AttributeTok{group\_by\_var =} \ConstantTok{NULL}\NormalTok{, }\AttributeTok{overall =} \ConstantTok{TRUE}\NormalTok{) \{}
\NormalTok{    data }\SpecialCharTok{\%\textgreater{}\%}
        \FunctionTok{select}\NormalTok{(}\FunctionTok{all\_of}\NormalTok{(vars), }\ControlFlowTok{if}\NormalTok{ (}\SpecialCharTok{!}\NormalTok{overall) }\FunctionTok{all\_of}\NormalTok{(group\_by\_var)) }\SpecialCharTok{\%\textgreater{}\%}
        \FunctionTok{gather}\NormalTok{(Variável, Valor, }\SpecialCharTok{{-}}\ControlFlowTok{if}\NormalTok{ (}\SpecialCharTok{!}\NormalTok{overall) group\_by\_var) }\SpecialCharTok{\%\textgreater{}\%}
\NormalTok{        \{}
            \ControlFlowTok{if}\NormalTok{ (overall) \{}
                \FunctionTok{group\_by}\NormalTok{(., Variável)}
\NormalTok{            \} }\ControlFlowTok{else}\NormalTok{ \{}
                \FunctionTok{group\_by}\NormalTok{(., Variável, }\SpecialCharTok{!!}\FunctionTok{sym}\NormalTok{(group\_by\_var))}
\NormalTok{            \}}
\NormalTok{        \} }\SpecialCharTok{\%\textgreater{}\%}
        \FunctionTok{summarise}\NormalTok{(}
            \AttributeTok{n =} \FunctionTok{n}\NormalTok{(),}
            \AttributeTok{missing =} \FunctionTok{sum}\NormalTok{(}\FunctionTok{is.na}\NormalTok{(Valor)),}
            \AttributeTok{normality =} \FunctionTok{if\_else}\NormalTok{(}\FunctionTok{shapiro.test}\NormalTok{(Valor)}\SpecialCharTok{$}\NormalTok{p.value }\SpecialCharTok{\textgreater{}}\NormalTok{ p.value,}
                \StringTok{"parametric"}\NormalTok{, }\StringTok{"non{-}parametric"}
\NormalTok{            ),}
            \AttributeTok{mean =} \FunctionTok{mean}\NormalTok{(Valor, }\AttributeTok{na.rm =} \ConstantTok{TRUE}\NormalTok{),}
            \AttributeTok{median =} \FunctionTok{median}\NormalTok{(Valor, }\AttributeTok{na.rm =} \ConstantTok{TRUE}\NormalTok{),}
            \AttributeTok{sd =} \FunctionTok{sd}\NormalTok{(Valor, }\AttributeTok{na.rm =} \ConstantTok{TRUE}\NormalTok{),}
            \AttributeTok{se =} \FunctionTok{sd}\NormalTok{(Valor, }\AttributeTok{na.rm =} \ConstantTok{TRUE}\NormalTok{) }\SpecialCharTok{/} \FunctionTok{sqrt}\NormalTok{(}\FunctionTok{length}\NormalTok{(Valor)),}
            \AttributeTok{iqr =} \FunctionTok{IQR}\NormalTok{(Valor, }\AttributeTok{na.rm =} \ConstantTok{TRUE}\NormalTok{)}
\NormalTok{        )}
\NormalTok{\}}
\end{Highlighting}
\end{Shaded}

\hypertarget{anuxe1lise-de-frequuxeancias-para-variuxe1veis-categuxf3ricas}{%
\subsubsection{Análise de Frequências para Variáveis
Categóricas}\label{anuxe1lise-de-frequuxeancias-para-variuxe1veis-categuxf3ricas}}

Aqui calcula-se as frequências absoluta e relativa das variáveis
categóricas por grupo. As variáveis de interesse são: Sexo, Mutação,
Pseudomonas\_cronico, Internação, DNAcat243,SPO2cat, VO2categ.

\begin{Shaded}
\begin{Highlighting}[]
\NormalTok{calc\_freqs }\OtherTok{\textless{}{-}} \ControlFlowTok{function}\NormalTok{(data, vars) \{}
\NormalTok{    data }\SpecialCharTok{\%\textgreater{}\%}
        \FunctionTok{select}\NormalTok{(Grupo, }\FunctionTok{all\_of}\NormalTok{(vars)) }\SpecialCharTok{\%\textgreater{}\%}
        \FunctionTok{gather}\NormalTok{(Variável, Valor, }\SpecialCharTok{{-}}\NormalTok{Grupo) }\SpecialCharTok{\%\textgreater{}\%}
        \FunctionTok{count}\NormalTok{(Grupo, Variável, Valor) }\SpecialCharTok{\%\textgreater{}\%}
        \FunctionTok{group\_by}\NormalTok{(Variável, Grupo) }\SpecialCharTok{\%\textgreater{}\%}
        \FunctionTok{mutate}\NormalTok{(}
            \AttributeTok{Freq\_rel =}\NormalTok{ n }\SpecialCharTok{/} \FunctionTok{sum}\NormalTok{(n),}
            \AttributeTok{Freq\_abs =}\NormalTok{ n}
\NormalTok{        ) }\SpecialCharTok{\%\textgreater{}\%}
        \FunctionTok{select}\NormalTok{(Variável, Grupo, Valor, Freq\_abs, Freq\_rel) }\SpecialCharTok{\%\textgreater{}\%}
        \FunctionTok{arrange}\NormalTok{(Variável, Grupo, Valor)}
\NormalTok{\}}
\end{Highlighting}
\end{Shaded}

\hypertarget{diferenuxe7as-entre-grupos-e-pelo-efeito-da-idade}{%
\subsubsection{Diferenças entre Grupos e pelo Efeito da
Idade}\label{diferenuxe7as-entre-grupos-e-pelo-efeito-da-idade}}

Aqui calcula-se as diferenças entre grupos e diferenças pelo efeito da
idade, \textgreater= ou \textless{} de 20 anos para cada uma das
seguintes variáveis de interesse: Idade, Dias\_internacao,
IMC\_absoluto, DNA\_corrigido15x, VEF1percprevist, SpO2final,
VO2mLkgminfinal.

\begin{Shaded}
\begin{Highlighting}[]
\NormalTok{calc\_diffs }\OtherTok{\textless{}{-}} \ControlFlowTok{function}\NormalTok{(data, vars, normality, p.value, group\_var) \{}
    \FunctionTok{map\_df}\NormalTok{(vars, }\ControlFlowTok{function}\NormalTok{(var) \{}
\NormalTok{        normality }\OtherTok{\textless{}{-}} \FunctionTok{filter}\NormalTok{(normality, Variável }\SpecialCharTok{==}\NormalTok{ var)}
\NormalTok{        is\_normal }\OtherTok{\textless{}{-}}\NormalTok{ normality}\SpecialCharTok{$}\NormalTok{shapiro\_p }\SpecialCharTok{\textgreater{}}\NormalTok{ p.value}
\NormalTok{        test\_result }\OtherTok{\textless{}{-}} \ControlFlowTok{if}\NormalTok{ (is\_normal) \{}
            \FunctionTok{t.test}\NormalTok{(data[[var]] }\SpecialCharTok{\textasciitilde{}}\NormalTok{ data[[group\_var]])}
\NormalTok{        \} }\ControlFlowTok{else}\NormalTok{ \{}
            \FunctionTok{wilcox.test}\NormalTok{(data[[var]] }\SpecialCharTok{\textasciitilde{}}\NormalTok{ data[[group\_var]], }\AttributeTok{conf.int =} \ConstantTok{TRUE}\NormalTok{)}
\NormalTok{        \}}
\NormalTok{        effect\_size }\OtherTok{\textless{}{-}} \ControlFlowTok{if}\NormalTok{ (is\_normal) \{}
            \FunctionTok{cohen\_d}\NormalTok{(data, var, group\_var)}
\NormalTok{        \} }\ControlFlowTok{else}\NormalTok{ \{}
            \FunctionTok{biserial}\NormalTok{(data, var, group\_var)}
\NormalTok{        \}}
        \FunctionTok{tibble}\NormalTok{(}
\NormalTok{            Variável }\OtherTok{=}\NormalTok{ var,}
            \AttributeTok{Teste =}\NormalTok{ test\_result}\SpecialCharTok{$}\NormalTok{method,}
            \AttributeTok{df =} \FunctionTok{ifelse}\NormalTok{(is\_normal, test\_result}\SpecialCharTok{$}\NormalTok{parameter, }\ConstantTok{NA}\NormalTok{),}
\NormalTok{            Estatística }\OtherTok{=}\NormalTok{ test\_result}\SpecialCharTok{$}\NormalTok{statistic,}
            \AttributeTok{p\_value =}\NormalTok{ test\_result}\SpecialCharTok{$}\NormalTok{p.value,}
\NormalTok{            Significância }\OtherTok{=} \FunctionTok{ifelse}\NormalTok{(test\_result}\SpecialCharTok{$}\NormalTok{p.value }\SpecialCharTok{\textless{}}\NormalTok{ p.value, }\StringTok{"*"}\NormalTok{, }\StringTok{"{-}"}\NormalTok{),}
            \AttributeTok{Effect\_Size =}\NormalTok{ effect\_size,}
            \AttributeTok{CI\_min =}\NormalTok{ test\_result}\SpecialCharTok{$}\NormalTok{conf.int[}\DecValTok{1}\NormalTok{],}
            \AttributeTok{CI\_max =}\NormalTok{ test\_result}\SpecialCharTok{$}\NormalTok{conf.int[}\DecValTok{2}\NormalTok{]}
\NormalTok{        )}
\NormalTok{    \}) }\SpecialCharTok{\%\textgreater{}\%}
        \FunctionTok{arrange}\NormalTok{(p\_value)}
\NormalTok{\}}
\end{Highlighting}
\end{Shaded}

\hypertarget{cuxe1lculo-de-tamanho-do-efeito}{%
\subsubsection{Cálculo de Tamanho do
Efeito}\label{cuxe1lculo-de-tamanho-do-efeito}}

Aqui calcula-se o tamanho do efeito para as diferenças entre grupos e
diferenças pelo efeito da idade, \textgreater= ou \textless{} de 20 anos
para cada uma das seguintes variáveis de interesse: Idade,
Dias\_internacao, IMC\_absoluto, DNA\_corrigido15x, VEF1percprevist,
SpO2final, VO2mLkgminfinal.

\begin{Shaded}
\begin{Highlighting}[]
\NormalTok{cohen\_d }\OtherTok{\textless{}{-}} \ControlFlowTok{function}\NormalTok{(data, var, group\_var) \{}
\NormalTok{    group1 }\OtherTok{\textless{}{-}}\NormalTok{ data }\SpecialCharTok{\%\textgreater{}\%}
        \FunctionTok{filter}\NormalTok{(data[[group\_var]] }\SpecialCharTok{==} \FunctionTok{unique}\NormalTok{(data[[group\_var]])[}\DecValTok{1}\NormalTok{]) }\SpecialCharTok{\%\textgreater{}\%}
        \FunctionTok{pull}\NormalTok{(var)}
\NormalTok{    group2 }\OtherTok{\textless{}{-}}\NormalTok{ data }\SpecialCharTok{\%\textgreater{}\%}
        \FunctionTok{filter}\NormalTok{(data[[group\_var]] }\SpecialCharTok{==} \FunctionTok{unique}\NormalTok{(data[[group\_var]])[}\DecValTok{2}\NormalTok{]) }\SpecialCharTok{\%\textgreater{}\%}
        \FunctionTok{pull}\NormalTok{(var)}
\NormalTok{    mean\_diff }\OtherTok{\textless{}{-}} \FunctionTok{mean}\NormalTok{(group1, }\AttributeTok{na.rm =} \ConstantTok{TRUE}\NormalTok{) }\SpecialCharTok{{-}} \FunctionTok{mean}\NormalTok{(group2, }\AttributeTok{na.rm =} \ConstantTok{TRUE}\NormalTok{)}
\NormalTok{    pooled\_sd }\OtherTok{\textless{}{-}} \FunctionTok{sqrt}\NormalTok{(((}\FunctionTok{length}\NormalTok{(group1) }\SpecialCharTok{{-}} \DecValTok{1}\NormalTok{) }\SpecialCharTok{*} \FunctionTok{var}\NormalTok{(group1, }\AttributeTok{na.rm =} \ConstantTok{TRUE}\NormalTok{) }\SpecialCharTok{+}\NormalTok{ (}\FunctionTok{length}\NormalTok{(group2) }\SpecialCharTok{{-}} \DecValTok{1}\NormalTok{) }\SpecialCharTok{*} \FunctionTok{var}\NormalTok{(group2, }\AttributeTok{na.rm =} \ConstantTok{TRUE}\NormalTok{)) }\SpecialCharTok{/}\NormalTok{ (}\FunctionTok{length}\NormalTok{(group1) }\SpecialCharTok{+} \FunctionTok{length}\NormalTok{(group2) }\SpecialCharTok{{-}} \DecValTok{2}\NormalTok{))}
\NormalTok{    cohen\_d }\OtherTok{\textless{}{-}}\NormalTok{ mean\_diff }\SpecialCharTok{/}\NormalTok{ pooled\_sd}
    \FunctionTok{return}\NormalTok{(cohen\_d)}
\NormalTok{\}}

\NormalTok{biserial }\OtherTok{\textless{}{-}} \ControlFlowTok{function}\NormalTok{(data, var, group\_var) \{}
\NormalTok{    data }\OtherTok{\textless{}{-}}\NormalTok{ data }\SpecialCharTok{\%\textgreater{}\%} \FunctionTok{mutate}\NormalTok{(}\AttributeTok{Rank =} \FunctionTok{rank}\NormalTok{(data[[var]], }\AttributeTok{na.last =} \StringTok{"keep"}\NormalTok{, }\AttributeTok{ties.method =} \StringTok{"average"}\NormalTok{))}
\NormalTok{    group1 }\OtherTok{\textless{}{-}}\NormalTok{ data }\SpecialCharTok{\%\textgreater{}\%}
        \FunctionTok{filter}\NormalTok{(data[[group\_var]] }\SpecialCharTok{==} \FunctionTok{unique}\NormalTok{(data[[group\_var]])[}\DecValTok{1}\NormalTok{]) }\SpecialCharTok{\%\textgreater{}\%}
        \FunctionTok{pull}\NormalTok{(Rank)}
\NormalTok{    group2 }\OtherTok{\textless{}{-}}\NormalTok{ data }\SpecialCharTok{\%\textgreater{}\%}
        \FunctionTok{filter}\NormalTok{(data[[group\_var]] }\SpecialCharTok{==} \FunctionTok{unique}\NormalTok{(data[[group\_var]])[}\DecValTok{2}\NormalTok{]) }\SpecialCharTok{\%\textgreater{}\%}
        \FunctionTok{pull}\NormalTok{(Rank)}
\NormalTok{    rbc }\OtherTok{\textless{}{-}}\NormalTok{ (}\FunctionTok{sum}\NormalTok{(group1) }\SpecialCharTok{/} \FunctionTok{length}\NormalTok{(group1) }\SpecialCharTok{{-}} \FunctionTok{sum}\NormalTok{(group2) }\SpecialCharTok{/} \FunctionTok{length}\NormalTok{(group2)) }\SpecialCharTok{/} \FunctionTok{length}\NormalTok{(data[[var]])}
    \FunctionTok{return}\NormalTok{(rbc)}
\NormalTok{\}}
\end{Highlighting}
\end{Shaded}

\hypertarget{correlauxe7uxf5es-entre-variuxe1veis}{%
\subsubsection{Correlações entre
Variáveis}\label{correlauxe7uxf5es-entre-variuxe1veis}}

Aqui calcula-se as correlações entre as variáveis de interesse: Idade,
Dias\_internacao, IMC\_absoluto, DNA\_corrigido15x, VEF1percprevist,
SpO2final, VO2mLkgminfinal. Ainda, extrai-se a variável
DNA\_corrigido15x para análise isolada.

\begin{Shaded}
\begin{Highlighting}[]
\NormalTok{calc\_corrs }\OtherTok{\textless{}{-}} \ControlFlowTok{function}\NormalTok{(data, vars, norms, p.value) \{}
    \FunctionTok{combn}\NormalTok{(vars, }\DecValTok{2}\NormalTok{, }\AttributeTok{simplify =} \ConstantTok{FALSE}\NormalTok{) }\SpecialCharTok{\%\textgreater{}\%}
        \FunctionTok{map\_df}\NormalTok{(}\ControlFlowTok{function}\NormalTok{(pair) \{}
\NormalTok{            norms1 }\OtherTok{\textless{}{-}} \FunctionTok{filter}\NormalTok{(norms, Variável }\SpecialCharTok{==}\NormalTok{ pair[}\DecValTok{1}\NormalTok{])}
\NormalTok{            norms2 }\OtherTok{\textless{}{-}} \FunctionTok{filter}\NormalTok{(norms, Variável }\SpecialCharTok{==}\NormalTok{ pair[}\DecValTok{2}\NormalTok{])}
\NormalTok{            is\_normal }\OtherTok{\textless{}{-}}\NormalTok{ norms1}\SpecialCharTok{$}\NormalTok{shapiro\_p }\SpecialCharTok{\textgreater{}}\NormalTok{ p.value }\SpecialCharTok{\&}\NormalTok{ norms2}\SpecialCharTok{$}\NormalTok{shapiro\_p }\SpecialCharTok{\textgreater{}}\NormalTok{ p.value}
\NormalTok{            corr\_result }\OtherTok{\textless{}{-}} \ControlFlowTok{if}\NormalTok{ (is\_normal) }\FunctionTok{cor.test}\NormalTok{(data[[pair[}\DecValTok{1}\NormalTok{]]], data[[pair[}\DecValTok{2}\NormalTok{]]],}
      \AttributeTok{method =} \StringTok{"pearson"}\NormalTok{) }\ControlFlowTok{else} \FunctionTok{cor.test}\NormalTok{(data[[pair[}\DecValTok{1}\NormalTok{]]], data[[pair[}\DecValTok{2}\NormalTok{]]], }\AttributeTok{method =} \StringTok{"spearman"}\NormalTok{)}
            \FunctionTok{tibble}\NormalTok{(}
\NormalTok{                Variável1 }\OtherTok{=}\NormalTok{ pair[}\DecValTok{1}\NormalTok{],}
\NormalTok{                Variável2 }\OtherTok{=}\NormalTok{ pair[}\DecValTok{2}\NormalTok{],}
                \AttributeTok{Teste =}\NormalTok{ corr\_result}\SpecialCharTok{$}\NormalTok{method,}
                \AttributeTok{df =}\NormalTok{ corr\_result}\SpecialCharTok{$}\NormalTok{parameter,}
\NormalTok{                Correlação }\OtherTok{=}\NormalTok{ corr\_result}\SpecialCharTok{$}\NormalTok{estimate,}
                \AttributeTok{p\_value =}\NormalTok{ corr\_result}\SpecialCharTok{$}\NormalTok{p.value,}
\NormalTok{                Significância }\OtherTok{=} \FunctionTok{ifelse}\NormalTok{(corr\_result}\SpecialCharTok{$}\NormalTok{p.value }\SpecialCharTok{\textless{}}\NormalTok{ p.value, }\StringTok{"*"}\NormalTok{, }\StringTok{"{-}"}\NormalTok{),}
                \AttributeTok{CI\_lower =}\NormalTok{ corr\_result}\SpecialCharTok{$}\NormalTok{conf.int[}\DecValTok{1}\NormalTok{],}
                \AttributeTok{CI\_upper =}\NormalTok{ corr\_result}\SpecialCharTok{$}\NormalTok{conf.int[}\DecValTok{2}\NormalTok{]}
\NormalTok{            )}
\NormalTok{        \})}
\NormalTok{\}}
\end{Highlighting}
\end{Shaded}

\hypertarget{apresentauxe7uxe3o-de-dados}{%
\subsubsection{Apresentação de
dados}\label{apresentauxe7uxe3o-de-dados}}

Define uma função de formatação dos resultados em formato de tabela.

\begin{Shaded}
\begin{Highlighting}[]
\NormalTok{custom\_kable }\OtherTok{\textless{}{-}} \ControlFlowTok{function}\NormalTok{(table\_data, ...) \{}
\NormalTok{    table\_output }\OtherTok{\textless{}{-}} \FunctionTok{kable}\NormalTok{(table\_data, ...) }
    \FunctionTok{return}\NormalTok{(table\_output)}
\NormalTok{\}}
\end{Highlighting}
\end{Shaded}

\newpage

\hypertarget{resultados}{%
\subsection{Resultados}\label{resultados}}

Abaixo seguem as tabelas com os resultados das análises descritas.

\hypertarget{teste-de-normalidade-1}{%
\subsubsection{Teste de Normalidade}\label{teste-de-normalidade-1}}

\begin{tabular}{l|r|r|r|l}
\hline
Variável & n & shapiro\_w & shapiro\_p & normality\\
\hline
Altura\_cm & 16 & 0.912 & 0.128 & parametric\\
\hline
CVFpercprevist & 16 & 0.899 & 0.076 & parametric\\
\hline
DNA\_corrigido15x & 16 & 0.931 & 0.251 & parametric\\
\hline
Dias\_ATB & 16 & 0.903 & 0.125 & parametric\\
\hline
Dias\_internacao & 16 & 0.753 & 0.001 & non-parametric\\
\hline
IMC\_absoluto & 16 & 0.909 & 0.112 & parametric\\
\hline
Idade & 16 & 0.894 & 0.066 & parametric\\
\hline
Peso\_kg & 16 & 0.906 & 0.100 & parametric\\
\hline
ReservaVentilatoria & 16 & 0.773 & 0.001 & non-parametric\\
\hline
SpO2final & 16 & 0.936 & 0.307 & parametric\\
\hline
VEF1percprevist & 16 & 0.862 & 0.021 & non-parametric\\
\hline
VEabsolutofinal & 16 & 0.952 & 0.524 & parametric\\
\hline
VO2mLkgminfinal & 16 & 0.970 & 0.845 & parametric\\
\hline
\end{tabular}

\hypertarget{estatuxedsticas-descritivas-1}{%
\subsubsection{Estatísticas
Descritivas}\label{estatuxedsticas-descritivas-1}}

Toda amostra

\begin{tabular}{l|c|c|l|r|r|r|r|r}
\hline
Variável & n & missing & normality & mean & median & sd & se & iqr\\
\hline
Altura\_cm & 16 & 0 & parametric & 158.9 & 163.0 & 19.1 & 4.8 & 21.2\\
\hline
CVFpercprevist & 16 & 0 & parametric & 65.1 & 53.3 & 25.4 & 6.3 & 42.5\\
\hline
DNA\_corrigido15x & 16 & 0 & parametric & 305.3 & 334.0 & 167.1 & 41.8 & 252.9\\
\hline
Dias\_ATB & 16 & 2 & parametric & 53.3 & 49.0 & 30.3 & 7.6 & 43.0\\
\hline
Dias\_internacao & 16 & 0 & non-parametric & 10.2 & 0.0 & 14.4 & 3.6 & 15.8\\
\hline
IMC\_absoluto & 16 & 0 & parametric & 19.8 & 19.6 & 2.8 & 0.7 & 2.8\\
\hline
Idade & 16 & 0 & parametric & 17.4 & 19.8 & 7.1 & 1.8 & 12.7\\
\hline
Peso\_kg & 16 & 0 & parametric & 51.3 & 55.0 & 14.9 & 3.7 & 17.5\\
\hline
ReservaVentilatoria & 16 & 0 & non-parametric & 36.8 & 43.7 & 19.9 & 5.0 & 15.1\\
\hline
SpO2final & 16 & 0 & parametric & 90.6 & 91.0 & 6.3 & 1.6 & 12.0\\
\hline
VEF1percprevist & 16 & 0 & non-parametric & 51.8 & 35.7 & 28.1 & 7.0 & 45.9\\
\hline
VEabsolutofinal & 16 & 0 & parametric & 44.2 & 45.0 & 15.9 & 4.0 & 28.5\\
\hline
VO2mLkgminfinal & 16 & 0 & parametric & 32.8 & 32.7 & 5.2 & 1.3 & 6.6\\
\hline
\end{tabular}

Por Categoria de Idade

\begin{tabular}{l|c|c|l|r|r|r|r|r|l}
\hline
Variável & Grupo & n & missing & normality & mean & median & sd & se & iqr\\
\hline
Altura\_cm & 1 & 8 & 0 & parametric & 147.1 & 148.8 & 19.2 & 6.8 & 34.5\\
\hline
Altura\_cm & 2 & 8 & 0 & parametric & 170.8 & 172.4 & 9.6 & 3.4 & 13.4\\
\hline
CVFpercprevist & 1 & 8 & 0 & parametric & 76.6 & 83.4 & 24.8 & 8.8 & 38.7\\
\hline
CVFpercprevist & 2 & 8 & 0 & parametric & 53.5 & 45.6 & 21.5 & 7.6 & 20.2\\
\hline
DNA\_corrigido15x & 1 & 8 & 0 & parametric & 221.8 & 177.5 & 132.3 & 46.8 & 236.1\\
\hline
DNA\_corrigido15x & 2 & 8 & 0 & parametric & 388.8 & 366.5 & 162.3 & 57.4 & 121.0\\
\hline
Dias\_ATB & 1 & 8 & 1 & parametric & 51.9 & 42.0 & 32.0 & 11.3 & 21.0\\
\hline
Dias\_ATB & 2 & 8 & 1 & parametric & 54.7 & 56.0 & 31.0 & 11.0 & 52.0\\
\hline
Dias\_internacao & 1 & 8 & 0 & non-parametric & 8.2 & 0.0 & 12.0 & 4.2 & 16.5\\
\hline
Dias\_internacao & 2 & 8 & 0 & non-parametric & 12.2 & 7.0 & 17.0 & 6.0 & 15.8\\
\hline
IMC\_absoluto & 1 & 8 & 0 & parametric & 19.6 & 17.8 & 4.0 & 1.4 & 4.4\\
\hline
IMC\_absoluto & 2 & 8 & 0 & parametric & 20.0 & 19.9 & 0.8 & 0.3 & 1.2\\
\hline
Idade & 1 & 8 & 0 & parametric & 15.5 & 13.2 & 7.3 & 2.6 & 11.4\\
\hline
Idade & 2 & 8 & 0 & parametric & 19.3 & 21.1 & 6.8 & 2.4 & 6.3\\
\hline
Peso\_kg & 1 & 8 & 0 & parametric & 44.1 & 42.6 & 17.6 & 6.2 & 29.9\\
\hline
Peso\_kg & 2 & 8 & 0 & parametric & 58.4 & 57.9 & 7.2 & 2.5 & 9.9\\
\hline
ReservaVentilatoria & 1 & 8 & 0 & non-parametric & 30.9 & 45.5 & 25.8 & 9.1 & 24.8\\
\hline
ReservaVentilatoria & 2 & 8 & 0 & parametric & 42.7 & 41.3 & 10.0 & 3.5 & 15.2\\
\hline
SpO2final & 1 & 8 & 0 & non-parametric & 93.8 & 97.0 & 6.5 & 2.3 & 4.8\\
\hline
SpO2final & 2 & 8 & 0 & parametric & 87.5 & 88.5 & 4.6 & 1.6 & 6.2\\
\hline
VEF1percprevist & 1 & 8 & 0 & parametric & 65.8 & 67.6 & 29.2 & 10.3 & 52.2\\
\hline
VEF1percprevist & 2 & 8 & 0 & non-parametric & 37.7 & 29.0 & 19.8 & 7.0 & 10.7\\
\hline
VEabsolutofinal & 1 & 8 & 0 & parametric & 40.5 & 37.0 & 17.8 & 6.3 & 27.9\\
\hline
VEabsolutofinal & 2 & 8 & 0 & parametric & 47.9 & 48.0 & 13.8 & 4.9 & 20.3\\
\hline
VO2mLkgminfinal & 1 & 8 & 0 & parametric & 32.6 & 32.7 & 3.9 & 1.4 & 6.0\\
\hline
VO2mLkgminfinal & 2 & 8 & 0 & parametric & 32.9 & 34.0 & 6.6 & 2.3 & 6.6\\
\hline
\end{tabular}

\hypertarget{anuxe1lise-de-frequuxeancias}{%
\subsubsection{Análise de
Frequências}\label{anuxe1lise-de-frequuxeancias}}

\begin{tabular}{l|r|r|r|r}
\hline
Variável & Grupo & Valor & Freq\_abs & Freq\_rel\\
\hline
DNAcat243 & 1 & 1 & 5 & 0.62\\
\hline
DNAcat243 & 1 & 2 & 3 & 0.38\\
\hline
DNAcat243 & 2 & 1 & 1 & 0.12\\
\hline
DNAcat243 & 2 & 2 & 7 & 0.88\\
\hline
Internacao & 1 & 1 & 3 & 0.38\\
\hline
Internacao & 1 & 2 & 5 & 0.62\\
\hline
Internacao & 2 & 1 & 4 & 0.50\\
\hline
Internacao & 2 & 2 & 4 & 0.50\\
\hline
Mutacao & 1 & 1 & 2 & 0.25\\
\hline
Mutacao & 1 & 2 & 3 & 0.38\\
\hline
Mutacao & 1 & 3 & 3 & 0.38\\
\hline
Mutacao & 2 & 1 & 6 & 0.75\\
\hline
Mutacao & 2 & 3 & 2 & 0.25\\
\hline
Pseudomonas\_cronico & 1 & 1 & 2 & 0.25\\
\hline
Pseudomonas\_cronico & 1 & 2 & 6 & 0.75\\
\hline
Pseudomonas\_cronico & 2 & 1 & 3 & 0.38\\
\hline
Pseudomonas\_cronico & 2 & 2 & 5 & 0.62\\
\hline
SPO2cat & 1 & 1 & 2 & 0.25\\
\hline
SPO2cat & 1 & 2 & 6 & 0.75\\
\hline
SPO2cat & 2 & 1 & 5 & 0.62\\
\hline
SPO2cat & 2 & 2 & 3 & 0.38\\
\hline
Sexo & 1 & 1 & 6 & 0.75\\
\hline
Sexo & 1 & 2 & 2 & 0.25\\
\hline
Sexo & 2 & 1 & 4 & 0.50\\
\hline
Sexo & 2 & 2 & 4 & 0.50\\
\hline
VO2categ & 1 & 1 & 3 & 0.38\\
\hline
VO2categ & 1 & 2 & 5 & 0.62\\
\hline
VO2categ & 2 & 1 & 2 & 0.25\\
\hline
VO2categ & 2 & 2 & 6 & 0.75\\
\hline
\end{tabular}

\hypertarget{diferenuxe7as-entre-grupos}{%
\subsubsection{Diferenças entre
Grupos}\label{diferenuxe7as-entre-grupos}}

\begin{tabular}{l|l|r|r|r|l|r|r|r}
\hline
Variável & Teste & df & Estatística & p\_value & Significância & Effect\_Size & CI\_min & CI\_max\\
\hline
VEF1percprevist & Wilcoxon rank sum exact test & NA & 56.00 & 0.01 & * & 0.38 & 5.09 & 60.38\\
\hline
DNA\_corrigido15x & Welch Two Sample t-test & 13.45 & -2.26 & 0.04 & * & -1.13 & -326.42 & -7.57\\
\hline
SpO2final & Welch Two Sample t-test & 12.63 & 2.23 & 0.04 & * & 1.11 & 0.17 & 12.33\\
\hline
Idade & Welch Two Sample t-test & 13.94 & -1.08 & 0.30 & - & -0.54 & -11.41 & 3.75\\
\hline
Dias\_internacao & Wilcoxon rank sum test with continuity correction & NA & 29.00 & 0.77 & - & -0.05 & -14.00 & 14.00\\
\hline
IMC\_absoluto & Welch Two Sample t-test & 7.57 & -0.27 & 0.79 & - & -0.13 & -3.77 & 2.99\\
\hline
VO2mLkgminfinal & Welch Two Sample t-test & 11.32 & -0.11 & 0.91 & - & -0.06 & -6.23 & 5.63\\
\hline
\end{tabular}

\hypertarget{diferenuxe7as-pelo-efeito-da-idade}{%
\subsubsection{Diferenças pelo Efeito da
Idade}\label{diferenuxe7as-pelo-efeito-da-idade}}

\begin{tabular}{l|l|r|r|r|l|r|r|r}
\hline
Variável & Teste & df & Estatística & p\_value & Significância & Effect\_Size & CI\_min & CI\_max\\
\hline
VEF1percprevist & Wilcoxon rank sum exact test & NA & 45.00 & 0.19 & - & 0.20 & -8.79 & 57.67\\
\hline
DNA\_corrigido15x & Welch Two Sample t-test & 12.10 & -0.83 & 0.42 & - & -0.42 & -254.01 & 113.40\\
\hline
IMC\_absoluto & Welch Two Sample t-test & 9.18 & -0.67 & 0.52 & - & -0.34 & -4.19 & 2.27\\
\hline
VO2mLkgminfinal & Welch Two Sample t-test & 12.04 & 0.37 & 0.72 & - & 0.19 & -4.86 & 6.86\\
\hline
Dias\_internacao & Wilcoxon rank sum test with continuity correction & NA & 30.00 & 0.86 & - & -0.03 & -14.00 & 14.00\\
\hline
\end{tabular}

\hypertarget{correlauxe7uxf5es-entre-variuxe1veis-1}{%
\subsubsection{Correlações entre
Variáveis}\label{correlauxe7uxf5es-entre-variuxe1veis-1}}

\begin{tabular}{l|l|l|r|r|r|l|r|r}
\hline
Variável1 & Variável2 & Teste & df & Correlação & p\_value & Significância & CI\_lower & CI\_upper\\
\hline
DNA\_corrigido15x & Idade & Pearson's product-moment correlation & 14 & 0.19 & 0.49 & - & -0.34 & 0.62\\
\hline
DNA\_corrigido15x & Dias\_internacao & Spearman's rank correlation rho & NA & 0.27 & 0.31 & - & NA & NA\\
\hline
DNA\_corrigido15x & IMC\_absoluto & Pearson's product-moment correlation & 14 & 0.00 & 0.99 & - & -0.50 & 0.49\\
\hline
DNA\_corrigido15x & VEF1percprevist & Spearman's rank correlation rho & NA & -0.46 & 0.08 & - & NA & NA\\
\hline
DNA\_corrigido15x & VO2mLkgminfinal & Pearson's product-moment correlation & 14 & -0.19 & 0.48 & - & -0.63 & 0.34\\
\hline
DNA\_corrigido15x & ReservaVentilatoria & Spearman's rank correlation rho & NA & 0.20 & 0.45 & - & NA & NA\\
\hline
Idade & Dias\_internacao & Spearman's rank correlation rho & NA & 0.21 & 0.43 & - & NA & NA\\
\hline
Idade & IMC\_absoluto & Pearson's product-moment correlation & 14 & 0.25 & 0.36 & - & -0.28 & 0.66\\
\hline
Idade & VEF1percprevist & Spearman's rank correlation rho & NA & -0.35 & 0.18 & - & NA & NA\\
\hline
Idade & VO2mLkgminfinal & Pearson's product-moment correlation & 14 & -0.01 & 0.98 & - & -0.50 & 0.49\\
\hline
Idade & ReservaVentilatoria & Spearman's rank correlation rho & NA & 0.13 & 0.62 & - & NA & NA\\
\hline
Dias\_internacao & IMC\_absoluto & Spearman's rank correlation rho & NA & -0.02 & 0.95 & - & NA & NA\\
\hline
Dias\_internacao & VEF1percprevist & Spearman's rank correlation rho & NA & -0.61 & 0.01 & * & NA & NA\\
\hline
Dias\_internacao & VO2mLkgminfinal & Spearman's rank correlation rho & NA & -0.58 & 0.02 & * & NA & NA\\
\hline
Dias\_internacao & ReservaVentilatoria & Spearman's rank correlation rho & NA & -0.47 & 0.06 & - & NA & NA\\
\hline
IMC\_absoluto & VEF1percprevist & Spearman's rank correlation rho & NA & -0.13 & 0.62 & - & NA & NA\\
\hline
IMC\_absoluto & VO2mLkgminfinal & Pearson's product-moment correlation & 14 & 0.18 & 0.50 & - & -0.35 & 0.62\\
\hline
IMC\_absoluto & ReservaVentilatoria & Spearman's rank correlation rho & NA & 0.02 & 0.94 & - & NA & NA\\
\hline
VEF1percprevist & VO2mLkgminfinal & Spearman's rank correlation rho & NA & 0.24 & 0.36 & - & NA & NA\\
\hline
VEF1percprevist & ReservaVentilatoria & Spearman's rank correlation rho & NA & 0.30 & 0.26 & - & NA & NA\\
\hline
VO2mLkgminfinal & ReservaVentilatoria & Spearman's rank correlation rho & NA & 0.03 & 0.92 & - & NA & NA\\
\hline
\end{tabular}

\hypertarget{exportauxe7uxe3o-dos-resultados}{%
\subsection{Exportação dos
Resultados}\label{exportauxe7uxe3o-dos-resultados}}

Os resultados foram exportados para arquivos CSV e excel para uso
posterior. Os arquivos são salvos na pasta \texttt{output}. A tabela
principal com os resultados finais está no arquivo ``results.xlsx'', com
cada análise em uma aba da planilha.

\hypertarget{conclusuxe3o}{%
\subsection{Conclusão}\label{conclusuxe3o}}

\hypertarget{acesso-ao-cuxf3digo-da-anuxe1lise-e-aos-dados}{%
\subsubsection{Acesso ao código da análise e aos
dados}\label{acesso-ao-cuxf3digo-da-anuxe1lise-e-aos-dados}}

Esse projeto, com o código R acima, assim como os dados associados podem
ser acessado no seguinte link:

\hypertarget{limitauxe7uxf5es}{%
\subsubsection{Limitações:}\label{limitauxe7uxf5es}}

A análise ainda carece de toda a parte de visualização dos resultados e
dos dados, a qual deveria incluir gráficos de barra, boxplot, matriz de
correlação, entre outros.

\end{document}
